% Copyright 2022 by Marek Rychly <rychly@fit.vut.cz>.
%
\documentclass[10pt,xcolor=pdflatex,dvipsnames,table,oneside]{book}
% babel and encoding
\usepackage[czech]{babel}
\usepackage[T1]{fontenc}
\usepackage[utf8]{inputenc}
\usepackage{lmodern}

\usepackage{csquotes}% correct/formal language-specific quotations
\usepackage{microtype}% character protrusion, font expansion, adjustment of interword spacing, additional kerning, tracking, etc.
\usepackage{hyperref}% hyper-refs in PDF

\author{
    František Koleček, \href{mailto:xkolec08@stud.fit.vut.cz}{xkolec08@stud.fit.vut.cz} \\
    Tomáš Moravčík, \href{mailto:xmorav41@stud.fit.vut.cz}{xmorav41@stud.fit.vut.cz} \\
    David Sladký, \href{mailto:xsladk07@stud.fit.vut.cz}{xsladk07@stud.fit.vut.cz}
    }
\title{Příprava dat a jejich popisná charakteristika}
\date{zima 2022}

\begin{document}

\pagenumbering{roman}

\hypersetup{pageanchor=false}% disable hyperref anchor to title page as maketitle enforce pagenumbering to arabic which colides the titlepage with the first arabic page below
\maketitle
\hypersetup{pageanchor=true}

\tableofcontents

\newpage% force page-break to start the page numbering on a new page
\pagenumbering{arabic}

% DOKUMENTACE K DRUHE CASTI DRUHEHO PROJEKTU
\chapter{Příprava datové sady}

\section{Zadání}
Datovou sadu obsahující informace o~tučňácích (\verb|penguins_lter.csv|) je třeba transformovat do podoby vhodné pro dolovací úlohu – klasifikace druhů tučňáků na základě ostatních atributů. Výstupem jsou dva nové datové soubory – \verb|A.csv| je vhodný pro metody vyžadující kategorické atributy a \verb|B.csv| je vhodný pro metody vyžadující numerické atributy.
\section{Nástroje}
Jako nástroj pro úpravu datové sady byl zvolen programovací jazyk Python s využitím knihovny Pandas. Tato knihovna obsahuje spoustu užitečných nástrojů pro zpracování souborů formátu csv a~pro následnou práci s~daty. Pandas není součástí základní instalace Pythonu, je třeba ji doinstalovat příkazem \verb|pip install pandas|. Skript implementující přípravu datové sady je uložen v souboru \verb|modify_data.py|. Skript vyžaduje tři vstupní argumenty – cestu k původní datové sadě, cestu k výstupní sadě s kategorickými atributy a cestu k výstupní sadě s numerickými atributy.\

Příklad spuštění:\

\verb|py ./src/modify_data.py ./dataset/penguins_lter.csv A.csv B.csv|\

Případně pomocí Makefile:\

\verb|make modify|\
\section{Postup}
Zpracování dat je rozděleno na několik dílčích kroků, některé jsou společně pro obě varianty výstupů.
\subsection{Odstranění irelevantních atributů}
Jedná se o~odstranění sloupců, které nejsou relevantní pro klasifikaci druhu tučňáka. V tomto případě se jedná například o~různé identifikátory – například identifikátor studia, číslo sběru a~individuální ID. Dále byly odstraněny informace o~snůškách vajec, region, ostrov (byl ponechán u~kategorické verze) a~komentář.
Byly ponechány informace o~fyzikálních vlastnostech tučňáků, jejich pohlaví a~druhu.
\subsection{Řešení chybějících hodnot}
V záznamech nebylo mnoho případů chybějících hodnot, proto byly ve většině případů odstraněny, především ty záznamy, ve kterých chyběly informace o~fyzikálních vlastnostech tučňáků, nebo jejich pohlaví. Ve zbytku dat bylo několik záznamů, kde chyběly informace o~složení krve. Tyto atributy byly doplněny mediánovou hodnotou příslušného sloupce.
\subsection{Řešení odlehlých hodnot}
V~jednom případě byla ve sloupci Sex chyba – byla zde byla tečka namísto obvyklých \verb|MALE| nebo \verb|FEMALE|. Tento záznam byl proto odstraněn. Krom tohoto případu nebyly v~datech žádné další významně odlehlé hodnoty.
\subsection{Kategorizace numerických hodnot}
Pro datovou sadu A byly sloupce obsahující numerické hodnoty převedeny na kategorické hodnoty – konkrétně na určité intervaly hodnot. Jedná se o~rozdělení dat do „košů“ (binning), kdy interval mezi minimální a maximální hodnotou sloupce je rozdělen na několik stejně velkých intervalů. V~tomto případě bylo použito dvacet košů. Touto metodou byly upraveny všechny sloupce obsahující numerické hodnoty. Výsledná datová sada je v~souboru \verb|A.csv|.
\subsection{Převod kategorických hodnot na numerické}
V případě datové sady B bylo zapotřebí provést převod kategorických dat na numerická data a normalizovat vhodné sloupce. Sloupec specifikující druh tučňáka obsahuje tři různé druhy, ty byly nahrazeny čísly 1 2 a 3. Obdobně byly hodnoty \verb|MALE| a \verb|FEMALE| ve sloupci \verb|Sex| nahrazeny za hodnoty 0 a 1. Na závěr byla provedena min-max normalizace u~dat popisující složení krve, konkrétně byly hodnoty převedeny na rozsah mezi 0 a 1. Díky této úpravě jsou ve výsledné datové sadě všechny hodnoty nezáporné. Výsledná datová sada je v~souboru \verb|B.csv|.
\end{document}
